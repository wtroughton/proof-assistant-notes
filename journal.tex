\environment notebook

\starttext

\chapter{Log}

\section{Monday, June 5, 2023}
Coq is an interactive proof assistant. Like Python, in which integrated development is an option, you can fully write a file and have it complete a proof or goal.

A list of some keywords: 
\startitemize[packed]
\item \type{Definition}
\item \type{Inductive}
\item \type{Type}
\stopitemize

Coq has two parts: Gallina, the programming language part and LTac, the functional language for manipulating goals using tactics. Why two languages? LTac is a meta-programming language to manipulate, generate Gallina terms and interact with the Coq system, but it is not the only one. Gallina is an implementation of the Calculus of Cumulative Inductive Constructions, which I interpret as a way to define programs as proofs as a purely functional PL.

\stoptext
